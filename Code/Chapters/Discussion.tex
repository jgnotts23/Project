% Chapter Template

\chapter{Discussion} % Main chapter title

\label{Discussion} % Change X to a consecutive number; for referencing this chapter elsewhere, use \ref{ChapterX}

\subsection{Machine learning and its effectiveness}
Training the CNN on the data provided by \cite{Hill2018} was very effective and allowed for very accurate classification of their data. Despite the high precision when presented with data from Belize, the model's precision took a significant hit when presented with data from the Osa Peninsula. This may be because, whilst the Belize acoustic data was also from a tropical forest in South America, it was from a different country with different fauna, weather, and soundscape. However, the model was still able to identify 246 gunshots which I manually checked for authenticity. This is a promising result, and with some optimisation may lead to greater precision and effectiveness for the use of neural networks and PAM going forward. Re-training the CNN on the manually-labelled data from the Osa Peninsula resulted in a significant increase in the number of 'gunshots' returned. This is likely to be indicative of lower precision; however, this cannot be stated as a certainty as the returned 'gunshots' were not manually validated. Re-training the CNN on a combined dataset of data from Belize and the Osa Peninsula resulted in even more 'gunshots' being identified. It is very likely that this is a result of increased numbers of false positives and in turn, lower precision, as this model classified approximately 25\% of all four second clips as containing a gunshot, which seems doubtful. Again, manual validation would be required to support this assertion. Interestingly, the fourth dataset used to train the CNN - composed of authenticated positives and previously identified false postives labelled as negatives from the Osa Peninsula data - returned about 4\% fewer 'gunshots'. This perhaps shows that highlighting to the CNN sounds that are similar to gunshots, such as a branch snapping, as being negative, improves the precision of the model. In this case, manual validation would also be required to support these claims. This strongly suggests that the primary factor in determining the effectiveness of a CNN model in studies such as this is the quality of the dataset used to train the model on. Going forward, the logical next step in this study area would be to increase the size of the dataset and to continue to retrain the model on increasingly large - and hopefully more accurate - data. This is likely the best way to improve the classification accuracy in the future.



\subsection{Spatio-temporal hunting patterns on the Osa Peninsula}
The temporal analyses carried out on the authenticated gunshot data returned some significant and interesting results. Firstly, certain weekdays proved  to be significant indicators of gunshot frequency, which may be due to reasons similar to those suggested by \cite{Astaras2017}, who highlighted the low frequency of gunshots on known market days. For \cite{Astaras2017}, this implied that the majority of the hunting taking place was so that the meat could be sold for profit, rather than taking place for sport or sustenance, and this may also be the case for the hunting taking place on the Osa Peninsula. Regardless of the reasons behind the variation, this type of data will undoubtedly be useful for rangers patrolling these national parks if they are able to focus their efforts on certain days of the week. \\

\noindent Furthermore, time of day proved to be a very significant indicator of gunshot frequency, with over three times as many gunshots detected in the morning (0500 - 0930) compared to during the night (2100 - 0300). This is a stark contrast to the findings of \cite{Astaras2017} who found the complete opposite, with far more gunshots being detected during the night. As this study has used very similar techniques to the study undertaken by \cite{Astaras2017}, there is presumably a difference between the two areas themselves causing the diverging trends. This difference may be patrolling frequency. \cite{Astaras2017} postulated that the increased hunting rate at night was due to regular ranger patrols during the day. Patrols in the national parks of the Osa Peninsula are said to be infrequent, and, when they do occur, only cover very small areas (personal communication, Jenna Lawson, 2019). This perhaps means that the patrols are an insufficient deterrent to hunting during the day, and the (probably low) risks of being caught during patrols are taken rather than hunting at night in forests that are treacherous and difficult to navigate. It should be noted, however, that despite much more regular and thorough patrolling in the study site of \cite{Astaras2017}, the rangers' efforts do not seem to reduce the amount of illegal hunting occurring, but merely shift it to a different time of day, meaning that it takes place at night rather than during the day. Thus, the findings of this study ought to be considered in the management strategies of protected forests in the future, possibly by establishing night-time patrols in addition to day-time patrols.
