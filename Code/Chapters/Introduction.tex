% Chapter Template

\chapter{Introduction} % Main chapter title

\label{Introduction}


\section{Biodiversity loss}

It is well documented that global biodiversity loss is accelerating. Anthropogenic factors are often touted as the leading cause, either directly through deforestation and hunting, or indirectly through climate change and the introduction of invasive species \citep{Chiarucci2011, Doherty2016, Newbold2015}. A meta-analysis by \cite{DeVos2015} gave an arguably conservative estimate of current biodiversity loss as being 1,000 times greater than the background 'natural' rate, with a magnitude of 10,000 times greater also being plausible \citep{Ceballos2015}. This alarming rate of loss not only affects the lost taxa themselves, but can also lead to extensive reduction in ecosystem multifunctionality, typically impacting poorer human communities \citep{Chiarucci2011, Allan2015, Fanin2018, Cardinale2012, Fanin2018}.\\

\noindent Tropical forests are no exception to this global trend. In addition to the adverse effects of deforestation on biodiversity, it is thought that other forms of anthropogenic disturbance, such as vehicles, hunting, and light pollution, can double the biodiversity loss \citep{Barlow2016}.

\section{Difficulties of measuring biodiversity and its loss}

Whilst it is universally agreed that biodiversity is being lost \citep{Cardinale2012, Ceballos2015, Fanin2018}, measuring this can be inconsistent and even erroneous \citep{Rocchini2018} for the following reasons. Firstly, only about 15\% of all species have been described, and therefore for the vast majority of species on Earth, there is no census data \citep{Chapman2009}. Secondly, of those that have been described, very little is known of their distribution, population size, ecology, and life histories, with many species only known by a single specimen \citep{Chapman2009}. Finally, of those species that have been described and studied in greater depth, there are inconsistencies in survey methods and often a lack of baseline measures to compare to \citep{TheRoy2003}.  The survey method deemed 'best' is often specific to a certain level of organisation and spatial scale of interest, such as satellite imagery and ground surveys for rainforest plant surveys. Despite the challenges and shortcomings associated with measuring biodiversity loss, it is indisputable that its acceleration is rapid, making urgent the development of programmes to assess and monitor biodiversity which are suitable for the answering of large-scale ecological questions \citep{Chiarucci2011, }.
Whilst it is universally agreed that biodiversity is being lost \citep{Cardinale2012, Ceballos2015, Fanin2018}, measuring this can be inconsistent and even erroneous \citep{Rocchini2018} for the following reasons. Firstly, only about 15\% of all species have been described, and therefore for the vast majority of species on Earth, there is no census data \citep{Chapman2009}. Secondly, of those that have been described, very little is known of their distribution, population size, ecology, and life histories, with many species only known by a single specimen \citep{Chapman2009}. Finally, of those species that have been described and studied in greater depth, there are inconsistencies in survey methods and often a lack of baseline measures to compare to \citep{TheRoy2003, Rocchini2018}.  The survey method deemed 'best' is often specific to a certain level of organisation and spatial scale of interest, such as satellite imagery and ground surveys for rainforest plant surveys \citep{Rocchini2018}. Despite the challenges and shortcomings associated with measuring biodiversity loss, it is indisputable that its acceleration is rapid, making urgent the development of programmes to assess and monitor biodiversity which are suitable for the answering of large-scale ecological questions \citep{Chiarucci2011, Rocchini2018}.


\section{Costa Rica and conservation}

Costa Rica is no exception to the aforementioned trend of biodiversity loss \citep{Hobinger2012}, with the Osa Peninsula being an area of focus for recent studies as it contains a mixture of protected, partially-protected, and unprotected land, including three national parks, Corcovado, Piedras Blancas and the Terreba-Sierpe wetlands  \citep{Lawson2019}. Whilst protected areas have benefited some species, others, such as the Geoffroy’s spider monkey, \textit{Ateles geoffroyi}, are struggling. This is due to their diet, need for mature trees, and need for large areas to roam, with typical home ranges of 4 km$^2$, which is increasingly limiting their range and may be isolating populations, in turn reducing their survival and genetic variability \citep{Chapman1989}. \textit{A. geoffroyi} is classified as endangered by the International Union for Conservation of Nature (IUCN) due to a 50\% reduction in numbers over the last 45 years \citep{Cuaron2008}. This reduction may be negatively impacting other species, as \textit{A. geoffroyi} is known to disperse the seeds of up to 150 tree species \citep{VanRoosmalen1985, Pacheco2000}. As well as habitat fragmentation, \textit{A. geoffroyi} is being subjected to hunting in both protected and non-protected areas. \cite{Aquino2013} found hunted populations of \textit{A. geoffroyi} in Peru were 70-80\% less dense than non-hunted populations. Admittedly, the monitoring and prevention of hunting in protected areas is often difficult in large reserves, due to the limited resources available to both rangers and conservationists. However, a recent study by \cite{Hill2018} demonstrated that gunshots can be detected with acoustic sensors up to 1km away from the source, opening up the possibility of a more effective and cheaper migration strategy.

\section{Passive acoustic monitoring and its advantages}

Passive acoustic monitoring (PAM) is becoming an increasingly popular method for large-scale biodiversity monitoring, primarily due to its relatively low cost \citep{Browning2017, Gibb2019}. This involves deploying sound recorders in an environment and having them record for days or weeks at a time to either track a vocal species directly or to use a vocal species as a proxy for another species or the ecosystem as a whole. Previously, methods such as PAM have been greatly limited by high implementation costs, a lack of digitisation, and low data storage capacity \citep{Merchant2015}. However, advances over the last 10-15 years have reduced the impact of these constraints dramatically \citep{Hill2018, Gibb2019}. One audio sensor in particular that has been developed recently in a collaborative project between the University of Oxford and University of Southampton, AudioMoth, is making PAM not only a viable option for monitoring biodiversity loss, but also one of the best methods available \citep{Hill2018}. \\

\noindent In addition to the direct monitoring of biodiversity, PAM can also be used to track other acoustics which may be relevant to conservation, such as gunshots, which are generally associated with illegal hunting, particularly in protected areas. \cite{Astaras2017} used PAM in a national park in Cameroon to successfully monitor the rates of hunting in the area. They found that most hunting (68.6\%) occurred at night when ranger patrols were minimal, and that there was more illegal activity during the week. \cite{Astaras2017} therefore argue that the hunting is for the illegal meat trade rather than for sustenance or sport, as the meat is gathered during the week for the Saturday market days. The cost of the PAM equipment was recorded by \cite{Astaras2017} as being quite high, which may limit the availability of its implementation in other national parks. However, the recent development of much cheaper audio sensors by \cite{Hill2018} may aid the spread of these techniques in conservation areas around the world. Digitisation has also made PAM a more viable survey mehod as it allows significantly longer recording times and records the data in a more appropriate format for computer analysis \citep{Hill2018, Gibb2019}.


\section{Machine learning}

Once audio data are collected, ecological information can be extracted manually or automatically. Manual extraction involves either auditory or visual inspection of the data and classification of the sounds, which naturally incurs some bias based on the skill of the person performing the analysis \citep{Heinicke2015}. This may be a viable option with a skilled ecologist and a small dataset. However, the latter is becoming increasingly rare with advancing technology, and therefore the need for automated techniques is growing rapidly. Fortunately, automated techniques are experiencing notable improvement in terms of both accuracy and efficiency, largely due to the use of machine learning \citep{Digby2013}. Most automated tools utilise supervised machine learning and related methods, including artificial neural networks \citep{Walters2012}, random forest \citep{ZamoraGutierrez2016}, Hidden Markov Models \citep{Zilli2014}, and support vector machines \citep{Heinicke2015}. These methods commonly use libraries of species calls or other sounds to facilitate detection when presented with new recordings. Currently, the low accuracy of these systems means that full automation is rare, and manual validation is often required \citep{Kalan2016}. However, new methods such as unsupervised feature extraction \citep{Stowell2014} and deep convolutional neural networks \citep{Goeau2016} can learn to classify directly from spectrogram data, often making them more robust and resistant to noise. At present, the main limitation for deep convolutional neural networks is large, clean datasets to train on.



\section{Aims}

\begin{enumerate}
  \item To use data provided by \cite{Hill2018} to train a deep convolutional neural network that can detect gunshots in acoustic data
  \item To investigate the effectiveness of using machine learning in cases such as this
  \item To identify the presence of any spatio-temporal patterns of hunting on the Osa Peninsula
\end{enumerate}
