\documentclass[11pt, titlepage]{article}

\usepackage[margin=2cm]{geometry} % change default margins
\linespread{1.5} % 1.5 spacing

\usepackage{natbib}
\usepackage{pgfgantt}


% Set font to Arial
\usepackage{helvet}
\renewcommand{\familydefault}{\sfdefault}

\title{Using acoustic monitoring and machine learning to 
automate the detection of illegal logging and hunting in 
Costa Rica}
\author{Jacob Griffiths\\
[1cm]{\small Supervisor: Dr Cristina Banks-Leite}\\
{\small Co-supervisor: Dr James Rosindell}}
\date{}
\begin{document}
  \maketitle


  \section*{Introduction}
    Passive audio monitoring is becoming a useful tool for 
    conservationists as it allows huge amounts of data 
    to be collected at an increasingly low cost \citep{Hill2018}. 
    The automation of the processing of this data through machine
    learning is improving the use to conservationists further
    as many human hours can be saved with similar or lower error 
    rates \citep{Kalan2015}.
    In Costa Rica, these techniques have already been used to 
    track endangered species like the Geoffroys spider monkey,
    \textit{Ateles geoffroyi}, resulting
    in a large audio dataset, predominantly collected by Jenna
    Lawson. If machine learning can be successfully applied to 
    this dataset, it would save a lot of conservationist time and
    potentially reduce the rates of illegal activity in protected
    areas of Costa Rica.



  \section*{Materials \& methods}
    The dataset for this project has been provided by Jenna
    Lawson. It consists of data obtained from over 200 different sites, 
    from fully protected areas, areas with limited protection, and 
    areas with no protection. Each site has at least one week 
    of audio data available for it and the devices used are the same 
    as those used by \cite{Hill2018}.
    A training dataset has also been provided by \cite{Hill2018}
    who used similar techniques to automatically detect gunshots
    in Belize. Training data for both chainsaw and dog barking
    noises will need to be sourced. 
    Python will be used to write the code as it has proved an intuitive
    and robust language for machine learning when packages such as 
    scikit-learn are used. Audio data will be converted to a 
    visual representation, a spectogram, which the algorithm
    will learn to find patterns in, rather than using the audio
    directly.


  \section*{Anticipated outcomes}
    I hope to create a fully-functioning machine learning 
    algorithm that can detect gunshot, chainsaw and dog barking 
    noises when presented with audio data from the rainforest,
    with the lowest error rate possible. I anticipate that 
    gunshot detection is the most likely to be successful as
    I have a training dataset already and in general, gunshots
    are quite a distinct noise to hear in a rainforest. Chainsaws 
    may be slightly harder to identify as they could be confused 
    with other motor sounds such as cars and motorbikes but it
    may be possible to train these other sounds out.



  \section*{Feasibility}
    A similar project was undertaken by Duncan last year, with 
    limited success. However, he didn't have a training dataset
    and this is a crucial aspect of creating a successful
    machine learning algorithm and makes this project a lot 
    more feasible. Furthermore, the study by \cite{Hill2018} has
    shown it can be done, for gunshots at least.


    \begin{ganttchart}[y unit title=0.5cm,
			y unit chart=0.75cm,
			vgrid,hgrid, 			
			title label anchor/.style={below=-1.6ex},
			title height=1.1,
			bar/.style={fill=blue!50},
			incomplete/.style={fill=green},
			bar height=.6]{1}{24}
			\gantttitle{Project timeline}{24}\\
			\gantttitle{April}{4}
			\gantttitle{May}{4}
			\gantttitle{June}{4}
			\gantttitle{July}{4}
			\gantttitle{August}{4}
			\gantttitle{September}{4}\\
			\ganttbar{Literature review}{1}{6} \\
			\ganttbar{Machine learning course}{1}{6} \\
			\ganttbar{Writing and training algorithms}{5}{12} \\
			\ganttbar{Evaluating algorithms}{10}{16} \\
			\ganttbar{Write up}{16}{20} \\
			\ganttbar{Presentation and viva}{21}{24}
			

		\end{ganttchart}


  \section*{Budget}
    At this moment in time, neither myself or my supervisors
    see any potential expenses for this project. If any software
    or data does require a fee, I will submit a request if and when
    necessary.




  \bibliographystyle{agsm}
  \bibliography{Project}

  \pagebreak

  \section*{Approval}
    Dr Cristina Banks-Leite \\
    26/04/19 \\
    \includegraphics{assinatura.jpg}

\end{document}